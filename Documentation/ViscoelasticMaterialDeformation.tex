\documentclass[12pt,english]{article}
\usepackage{amsfonts,babel,amssymb,amsmath,amsthm}
\usepackage{graphicx}
\usepackage{framed}
\usepackage{mathabx}

%\usepackage[notcite,notref]{showkeys}
\usepackage{float} % to use \begin{figure}[H], which inserts the figure wherever you want
\usepackage[utf8]{inputenc} % gives extra characters

\evensidemargin 0cm \oddsidemargin 0cm \setlength{\topmargin}{-1cm}
\setlength{\textheight}{23truecm} \textwidth 16truecm

\newtheorem{proposition}{Proposition}[section]
\newtheorem{corollary}[proposition]{Corollary}
\newtheorem{lemma}[proposition]{Lemma}
\newtheorem{theorem}[proposition]{Theorem}

\newtheorem{remark}{Remark}[section]

\numberwithin{equation}{section}

% Generic 
\newcommand{\ds}{\displaystyle}
\newcommand{\jump}[1]{\llbracket#1\rrbracket}
\newcommand{\ave}[1]{\{\!\!\{#1\}\!\!\}}
\newcommand{\energy}[1]{|\!|\!|#1|\!|\!|}
\newcommand{\bs}[1]{\boldsymbol{#1}}
\newcommand{\R}{{\mathbb R}}
\newcommand{\C}{\mathbb C}
\newcommand{\Ndof}{N_{\mathrm{dof}}}


% Latex shortcuts
\newcommand{\bff}{\mathbf}
\newcommand{\mb}{\mathbf}
\newcommand{\mc}{\mathcal}
\newcommand{\mr}{\mathrm}

% New math
\renewcommand{\Re}{\mathrm{Re}\,}


% HEADMATTER

\title{Viscoelastic material deformation}

\author{Hasan Eruslu \\
University of Delaware}

\date{\today}

\begin{document}

\maketitle


\section{Fractional Zener model} \label{sec:2.1}
Our goal is computing a FEM approximation of the viscoelastic wave propagation problem with Fractional Zener model: for $t \geq 0$
\begin{subequations} \label{eq:2.0}
	\begin{alignat}{6}
	-\mathrm{div}\,\boldsymbol\sigma(t)+\rho\, \ddot{ \mathbf u} (t) &=\mathbf f(t)
	&\qquad &\mbox{in $\Omega$},\\
	\mathbf u(t) &=\mathbf u_D(t) &\qquad&\mbox{on $\Gamma_D$},\\
	\boldsymbol\sigma(t)\,\boldsymbol\nu &=\boldsymbol\sigma_N(t)\,\boldsymbol\nu
	&\qquad&\mbox{on $\Gamma_N$},
	\end{alignat}
where the stress-strain relation can be written in Laplace domain as
	\begin{align} \label{eq:2.0d}
		\mb S(s) &=\mu(s) \boldsymbol\varepsilon(\mathbf U(s))+\lambda(s) (\nabla\cdot\mathbf U(s))\mathrm I \qquad \forall s \in \C_+ : = \{ s \in \C : \Re s > 0 \}, 
	\end{align}
with
	\begin{align}
		\mu(s) = \frac{m_\mu + b_\mu s^{\nu_\mu}}{1 + a s^{\nu_\mu}}, \qquad 
		\lambda(s) = \frac{m_\lambda + b_\lambda s^{\nu_\lambda}}{1 + a s^{\nu_\lambda}}
	\end{align}
for $m_\mu,b_\mu,\nu_\mu,m_\lambda,b_\lambda,\nu_\lambda \in L^\infty(\Omega)$ strictly positive satisfying
	\[
		a m_\mu \leq b_\mu, \qquad a m_\lambda \leq b_\lambda.
	\] 
Here we assume homogeneous initial conditions
	\begin{align}
		\mb u (0) = \mb 0, \qquad \dot{\mb u}(0) = \mb 0.
	\end{align}
\end{subequations}

\subsection{Parameter details}
\begin{enumerate}
\item
$\rho$: Mass density function can be heterogenous. Name in the simulation:  {\tt rho} 
\item
$m_\mu,b_\mu,\nu_\mu,m_\lambda,b_\lambda,\nu_\lambda$: Viscoelastic parameters can be heterogenous. Corresponding names in the simulation: {\tt m\_mu, b\_mu, nu\_mu, m\_lam, b\_lam, nu\_lam}
\end{enumerate}


\section{Solver details}
Viscoelasticity solver computes the FEM coefficient vector of the displacement and the stress at DOF. This is done via the following steps.
\begin{enumerate}
\item
Sample the forcing term, Dirichlet and Neumann data at the time steps. This is done in a parallel loop using a helper function. Depending on the time stepping method, we sample a little differently:
\begin{enumerate}
\item
	\textbf{Sampling for the CQ method.} We create $3\Ndof \times 1$ vectors 
			\[
				\mb f^n, \quad \mb u^n_{h,D}, \quad \mb t^n, \qquad n=1,\ldots,N_t
			\]
		corresponding to load, Dirichlet and Neumann approximations at time steps $t_n = n \Delta$ (uniform time steps). Then we combine all these samples and construct the $6 \Ndof \times N_t$ matrix $\mr X$ in the following way
		\begin{align*}
		\mr X = \left[ \begin{array}{c|c|c|c}
			&	 & 		& 	\\
		X_1  & X_2 & \cdots & X_{N_t}\\
			&	 & 		& 	\\		
		\end{array} \right],
		\end{align*}
	where each column is
		\begin{align*}
			X_n = \left[ \begin{array}{c}
			\mb u^n_{h,D} \\[1em]
			\mb f^n + \mb t^n
			\end{array} \right ] \qquad n = 1, \ldots, N_t.
		\end{align*}
\item
	\textbf{Sampling for the RKCQ method.} We run a parallel loop over time steps $n=1,\ldots,N_t - 1$, and a serial loop over stages $\ell = 1,\ldots, S$ to create $3\Ndof \times 1$ vectors
			\[
				\mb f^{n,\ell}, \quad \mb u^{n,\ell}_{h,D}, \quad \mb t^{n,\ell},
			\]
		corresponding to load, Dirichlet and Neumann approximations at time steps 
			\[
				 t_{n,\ell} = (n  +  c_\ell) \Delta, \qquad n = 1,\ldots,N_t, \quad \ell = 1,\ldots,S.
			\]
		Here the index $\ell$ corresponds to the variable {\tt ns} in the code. The vector $\mb c$, on the other hand, is from the Butcher tableau and equal to the row sums of the $S \times S$ matrix $\mathbf A$. This matrix is an optional input if one wants to use Runge-Kutta as  time-stepping method.
		
		Now we first create $6 \Ndof \times N_t$ matrices
		\begin{align*}
		\mr X^\ell = \left[ \begin{array}{c|c|c|c}
			&	 & 		& 	\\
		X_1^\ell  & X_2^\ell & \cdots & X_{N_t}^\ell\\
			&	 & 		& 	\\		
		\end{array} \right] \qquad \ell = 1, \ldots, S
		\end{align*}
	where each column is
		\begin{align*}
			X^\ell_n = \left[ \begin{array}{c}
			\mb u^{n,\ell}_{h,D} \\[1em]
			\mb f^{n,\ell} + \mb t^{n,\ell}
			\end{array} \right ] \qquad n = 1,\ldots,N_t, \quad \ell = 1,\ldots,S.
		\end{align*}
	Then we vertically stack matrices $\mr X^\ell$ for $\ell = 1,\ldots,S$
		\begin{align*}
			\mr X = \left[ \begin{array}{c}
			\mr X^1 \\[1em]
			\vdots \\[1em]
			\mr X^S
			\end{array} \right ],
		\end{align*}
	to obtain a single $6 \Ndof S \times N_t$ matrix.

\end{enumerate}
	
\item 
We construct the transfer function $\mc F_h$ in the Laplace domain such that
\begin{align*}
		\mc F_h(s,\mb u_D,\mb f,\bs \sigma_N) = (\mb u^h, \bs \sigma^h)
	\end{align*}
is FEM approximation of
	\begin{alignat*}{6}
	-\mathrm{div}\,\boldsymbol\sigma+s^2\,\rho\, \mathbf u &=\mathbf f
	&\qquad &\mbox{in $\Omega$},\\
	\mathbf u &=\mathbf u_D &\qquad&\mbox{on $\Gamma_D$},\\
	\boldsymbol\sigma\,\boldsymbol\nu &=\boldsymbol\sigma_N\,\boldsymbol\nu
	&\qquad&\mbox{on $\Gamma_N$}.
	\end{alignat*}
\item
Using the transfer function $\mc F_h$ and sampled data $\mr X$ we perform an all-at-once CQ time stepping.
\item
We finally output the FEM approximation of displacement and post-processed stress 
	\[
		\mb u^n_h, \quad \bs \sigma^n_h
	\]
at the time steps $t_n$ for $n = 1, \cdots, N$. 
\end{enumerate}

\end{document}